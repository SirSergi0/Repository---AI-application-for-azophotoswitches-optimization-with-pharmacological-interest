\documentclass[11pt]{article}
\usepackage[utf8]{inputenc}
%----------FONT AND GEOMETRY DEFINITION----------%
\usepackage[a4paper, hmargin=3.0cm, vmargin=3.0cm]{geometry}
\voffset = -0.75cm
\usepackage{arev}
\usepackage{helvet}
\renewcommand{\rmdefault}{\sfdefault}
\usepackage[math]{blindtext}
\usepackage{amsmath,amsthm,amsfonts,amssymb,amscd, esint}
%---------SPACING----------%
\usepackage{setspace}
\setstretch{1.5}

\usepackage[english]{babel}

%-------------LINKS AND LOOKS-----------%
\usepackage{fullpage}
\usepackage{afterpage}
\usepackage{lastpage}
\usepackage{fancyhdr}
\usepackage{forest}
\usepackage{graphicx}
\usepackage{xcolor}
\usepackage{hyperref}
\usepackage{pgfplots}
\usepackage{xurl}
\usepackage{pdflscape}
%-----------------FIGURES AND TABLES----------------%
\usepackage{array}
%\usepackage{wrapfig}
\usepackage{caption}
\DeclareCaptionType{scheme}
%\usepackage{subcaption}
\usepackage{tabularx, booktabs, makecell, multicol, multirow, bigstrut} %for excel2latex
\usepackage{float}

%-----------------CHEMICAL FIGURES AND FORMULAS--------------------%
\usepackage{chemfig}
\usepackage{chemformula}

%----------BIBLIOGRAPHY----------%
%file with extnsion .bib is required (generated by Mendeley for example) called references in the folder where the .tex file is compiled
\usepackage[backend=bibtex, style = chem-acs]{biblatex}
\usepackage{csquotes}
%in case of writing the thesis in catalan or spanish:
%\usepackage[autostyle=false, style=spanish]{csquotes}

% check out the document types in the .bib file that biblatex can cite in: https://mirror.las.iastate.edu/tex-archive/macros/latex/contrib/biblatex-contrib/biblatex-chem/chem-acs.bbx

%---max amount of bibliography names
\ExecuteBibliographyOptions{maxnames=10}
%----
%-----article issue between parenthesis after the volume number - citation style modification
\DeclareFieldFormat[article]{issue}{\mkbibparens{#1}}
\renewbibmacro*{volume+number+eid}{%
  \printfield{volume}%
  \setunit{\space}%
  \printfield{issue}%
  \setunit{\addcomma\space}%
  \printfield{eid}}
%---------------------------------------------------------
\addbibresource{references.bib}

%----------TITLE AND PAGE FORMATTING----------%
\newcommand\NomComplet{Sergio Castañeiras Morales}
\newcommand\NIU{1598456}
\newcommand\NomProfeA{Dr. Miquel Moreno Ferrer}
\newcommand\NomProfeB{Dr. Àngels González Lafont}
\newcommand\CodiTFG{TFG2425\_045}
\newcommand\TitolTFG{AI Application for Azophotoswitches Optimization with Pharmacological Interest}
\newcommand\NomCentre{Departament de Quimica } %cal posar un espai al final
\newcommand\NomGrau{Quimica}
\newcommand\Mes{Juny } %cal espai al final
\newcommand\Any{2025}


\fancypagestyle{sumari}{
\fancyhf{}
\rhead{\TitolTFG}
\renewcommand{\headrulewidth}{0pt}%
\renewcommand{\footrulewidth}{0pt}%
}

\fancypagestyle{afteres}{
\fancyhf{}
\headheight 0 cm
\headsep 1.725cm
\rhead{\TitolTFG}
\cfoot{\thepage}
\renewcommand{\headrulewidth}{0pt}%
\renewcommand{\footrulewidth}{0pt}%
}

\pagestyle{fancy}
\fancyhf{}
\headheight 0 cm
\headsep 1.725cm
\rhead{\TitolTFG}
\cfoot{\thepage}
\renewcommand{\headrulewidth}{0pt}%
\renewcommand{\footrulewidth}{0pt}%

%----------METADATA------------%

\hypersetup{%
  colorlinks=false,
  linkcolor=black,
  linkbordercolor={1.0 1.0 1.0},
  breaklinks = true,
  citebordercolor={1.0 1.0 1.0},
  hidelinks = true,
  pdftitle  = \TitolTFG,
  pdfauthor = \NomComplet,
  pdfsubject= \CodiTFG,
  pdfcreator= \NomComplet,
}

%----------ABBREVIATIONS LIST----------%
\usepackage{glossaries}
\usepackage{glossary-longragged}

\makenoidxglossaries

\newacronym{cox2}{COX-2}{Cyclooxygenase-2}
\newacronym{nsaid}{NSAID}{Non-Steroidal Anti-Inflammatory Drug}
\newacronym{ptgs2}{PTGS2}{Prostaglandin-endoperoxide synthase 2}
\newacronym{ai}{AI}{Artificial intelligence}
\newacronym{ml}{ML}{Machine Learning}
\newacronym{rf}{RF}{Random Forest}
\newacronym{id}{ID}{Identificator}
\newacronym{ic50}{IC$_{50}$}{Half Maximal Inhibitory Concentration}
\newacronym{ic90}{IC$_{90}$}{90 Percent Inhibitory Concentration}
\newacronym{ic99}{IC$_{99}$}{99 Percent Inhibitory Concentration}
\newacronym{tp}{TP}{True Posive}
\newacronym{fp}{FP}{False Posive}
\newacronym{tn}{TN}{True Negative}
\newacronym{fn}{FN}{False Negative}

\glsaddall[types=\acronymtype] % to add all the acronyms, cited or not in the text


%----------ENVIRONMENTS------------------%
\newtheorem{definition}{Definition}

%----------DOCUMENT COMPILATION----------%
\begin{document}
\newgeometry{bottom = 1cm}
\begin{titlepage}

\center % Center everything on the page

%	LOGO SECTION
%----------------------------------------------------------------------------------------

\includegraphics[width = 6.1cm]{GeneralSources/Logo_uab.png}\\[1.76cm] % Include a department/university logo - this will require the graphicx package

% TITLE SECTION
%----------------------------------------------------------------------------------------

\textbf{\LARGE Facultat de Ciències}\\[3.53cm] % Major heading such as course name
%\rule{.1pt}{5cm}
\begin{flushright}
\begin{tabular}{r | p{.5\textwidth}}
  {\Large Treball de} & {\LARGE \CodiTFG}\\
  {\LARGE fi de grau}  & {\LARGE \TitolTFG}
\end{tabular}
\end{flushright}\vspace{5.06cm}


%----------------------------------------------------------------------------------------
%	AUTHOR SECTION
%----------------------------------------------------------------------------------------


\begin{tabular}{p{.48\textwidth} p{.48\textwidth}}
\large
Direcció: & Alumne: \\
\NomProfeA & \NomComplet\\
\NomProfeB & NIU: \\
 & \NIU
\end{tabular}
\vspace{0.5cm}


%----------------------------------------------------------------------------------------
%	DATE SECTION
%----------------------------------------------------------------------------------------

{\Large \Mes \Any}\\[0.5cm] % Date, change the \today to a set date if you want to be precise

%----------------------------------------------------------------------------------------
Treball de fi de grau realitzat al \NomCentre i presentat a la\\ Facultat de Ciancies\\ de la Universitat Autnòma de Barcelona per a l'obtenció del Grau en \NomGrau

%\vfill % Fill the rest of the page with whitespace

\end{titlepage}
\restoregeometry
\pagenumbering{Roman}
%\afterpage{\aftergroup\restoregeometry}
\thispagestyle{empty}
\mbox{} 
\newpage
\thispagestyle{empty}
\vspace*{\fill} % Push content to vertical center
\begin{flushright}
    \emph{“The dumbest people I know are those who know it all.”}\\[1em]
    \textbf{Malcolm S. Forbes}
\end{flushright}
\vspace*{\fill} 
\newpage
\thispagestyle{empty}
\mbox{} 
\newpage

\thispagestyle{afteres}
\setcounter{page}{1}
\section*{Resum analític}\par
L'intel·ligència artificial es presenta com una de les revolucions del segle XXI. En particular, el sector de la química computacional està sent profundament sacsejat per aquesta revolució. Aprofitant la inèrcia i l'interès creixent en aquest camp, aquest treball pretén aplicar diferents models d'intel·ligència artificial en l'estudi d'una proteïna d'especial interès per a la nostra salut, la Ciclooxigenasa-2 (COX-2).

La \emph{prostaglandin-endoperoxide sinthasa 2} (PTGS2), també coneguda com COX-2, és una proteïna que, en circumstàncies normals, acostuma a romandre inactiva \cite{Cox2Location}, llevat de la seva expressió durant processos inflamatoris. Així mateix, la manca de retorn a nivells baixos d'expressió després de la inflamació ha estat relacionada amb l'aparició de diferents formes de càncer \cite{DefinitionCOX2CancerDictionary}. Aquest fet ha convertit la COX-2 en objecte d'estudi de nombroses investigacions científiques \cite{Cox2InhibitorsReview}, fet que la fa un punt de partida idoni per al desenvolupament d'algoritmes de \emph{Machine Learning}, ja que disposa d'una gran quantitat de dades per entrenar els models i contrastar els resultats.

L'objectiu principal del projecte és el desenvolupament d'un programari generador d'IAs capaces de predir la concentració d'inhibició al 50\% (IC$_{50}$) per a la COX-2\footnote{En realitat, el programari funciona per a qualsevol proteïna amb entrada a la base de dades de ChEMBL \cite{ChemblDatabase}, malgrat que l'objecte d'estudi és la COX-2.} amb la màxima precisió possible. Per fer-ho, s'extreuen totes les dades de molècules conegudes amb un potencial d'inhibició establert per a la COX-2. Després d'un filtratge configurable per l'usuari, es calculen 5.900 descriptors químics per a cadascuna de les entrades amb el programari AlvaDesk \cite{MauriMolecularDescriptorsBook}\cite{AlvaDescSecondPaper}. Seguidament, una part de les dades s'utilitza per entrenar models de \emph{Random Forest} (RF) \cite{MachineLearningRandomForest}, mentre que la resta es reserva per validar la precisió de les prediccions.

Cal remarcar la principal hipòtesi que sustenta aquest procés i el projecte en general: \emph{Existeix una combinació (o diverses combinacions) de descriptors químics directament relacionada amb el potencial d'inhibició de la proteïna}. Malgrat que aquesta afirmació pugui semblar natural, el cost computacional associat és immens. Tot i així, la precisió de les prediccions dels models apunta a la validesa d'aquesta hipòtesi, si bé continua essent una conjectura per manca d'una prova definitiva.

Finalment, els models es fan servir per predir l'IC$_{50}$ de 50 \emph{azophotoswitches} dels quals es tenen dades sobre l'energia lliure d'acoblament proporcionades pel Departament de Química Física de la UAB \cite{UAB_ComputationalChemistry}. L'anàlisi estadístic de les prediccions reflecteix una clara correlació entre ambdues quantitats, fet que reforça la hipòtesi del projecte.
\newpage

\section*{Resumen analítico}
La inteligencia artificial se presenta como una de las revoluciones del siglo XXI. En particular, el sector de la química computacional está siendo profundamente sacudido por esta revolución. Aprovechando la inercia y el interés creciente en este campo, este trabajo pretende aplicar diferentes modelos de inteligencia artificial en el estudio de una proteína de especial interés para nuestra salud, la Ciclooxigenasa-2 (COX-2).

La \emph{prostaglandin-endoperoxide sinthasa 2} (PTGS2), también conocida como COX-2, es una proteína que, en circunstancias normales, suele permanecer inactiva \cite{Cox2Location}, excepto por su expresión durante procesos inflamatorios. Asimismo, la falta de retorno a niveles bajos de expresión después de la inflamación ha sido relacionada con la aparición de diferentes formas de cáncer \cite{DefinitionCOX2CancerDictionary}. Este hecho ha convertido a la COX-2 en objeto de estudio de numerosas investigaciones científicas \cite{Cox2InhibitorsReview}, lo que la convierte en un punto de partida idóneo para el desarrollo de algoritmos de \emph{Machine Learning}, ya que dispone de una gran cantidad de datos para entrenar los modelos y contrastar los resultados.

El objetivo principal del proyecto es el desarrollo de un software generador de IAs capaces de predecir la concentración de inhibición al 50\% (IC$_{50}$) para la COX-2\footnote{En realidad, el software funciona para cualquier proteína con entrada en la base de datos de ChEMBL \cite{ChemblDatabase}, aunque el objeto de estudio es la COX-2.} con la máxima precisión posible. Para ello, se extraen todos los datos de moléculas conocidas con un potencial de inhibición establecido para la COX-2. Tras un filtrado configurable por el usuario, se calculan 5.900 descriptores químicos para cada una de las entradas con el software AlvaDesk \cite{MauriMolecularDescriptorsBook}\cite{AlvaDescSecondPaper}. Seguidamente, una parte de los datos se utiliza para entrenar modelos de \emph{Random Forest} (RF) \cite{MachineLearningRandomForest}, mientras que el resto se reserva para validar la precisión de las predicciones.

Cabe remarcar la principal hipótesis que sustenta este proceso y el proyecto en general: \emph{Existe una combinación (o varias combinaciones) de descriptores químicos directamente relacionada con el potencial de inhibición de la proteína}. Aunque esta afirmación pueda parecer natural, el coste computacional asociado es inmenso. Aun así, la precisión de las predicciones de los modelos apunta a la validez de esta hipótesis, si bien sigue siendo una conjetura por falta de una prueba definitiva.

Finalmente, los modelos se utilizan para predecir el IC$_{50}$ de 50 \emph{azophotoswitches}, de los cuales se tienen datos sobre la energía libre de acoplamiento proporcionados por el Departamento de Química Física de la UAB \cite{UAB_ComputationalChemistry}. El análisis estadístico de las predicciones refleja una clara correlación entre ambas cantidades, lo que refuerza la hipótesis del proyecto.

\newpage

\section*{Analytical abstract}
Artificial intelligence is emerging as one of the revolutions of the 21st century. In particular, the field of computational chemistry is being profoundly shaken by this revolution. Taking advantage of the momentum and growing interest in this field, this work aims to apply different artificial intelligence models to the study of a protein of special interest to our health, Cyclooxygenase-2 (COX-2).

The \emph{prostaglandin-endoperoxide synthase 2} (PTGS2), also known as COX-2, is a protein that, under normal circumstances, tends to remain inactive \cite{Cox2Location}, except for its expression during inflammatory processes. Likewise, the failure to return to low expression levels after inflammation has been linked to the onset of various forms of cancer \cite{DefinitionCOX2CancerDictionary}. This fact has made COX-2 the subject of numerous scientific investigations \cite{Cox2InhibitorsReview}, making it an ideal starting point for the development of \emph{Machine Learning} algorithms, as it provides a large amount of data for training models and validating results.

The main objective of the project is to develop software capable of generating AIs that can predict the 50\% inhibition concentration (IC$_{50}$) for COX-2\footnote{In reality, the software works for any protein with an entry in the ChEMBL database \cite{ChemblDatabase}, although the study focuses on COX-2.} with the highest possible accuracy. To achieve this, all known molecular data with an established inhibition potential for COX-2 are extracted. After a user-configurable filtering process, 5,900 chemical descriptors are calculated for each entry using the AlvaDesk software \cite{MauriMolecularDescriptorsBook}\cite{AlvaDescSecondPaper}. Subsequently, part of the data is used to train \emph{Random Forest} (RF) models \cite{MachineLearningRandomForest}, while the rest is reserved to validate the accuracy of the predictions.

It is important to highlight the main hypothesis that underpins this process and the project as a whole: \emph{There exists a combination (or multiple combinations) of chemical descriptors that are directly related to the inhibition potential of the protein}. While this statement may seem intuitive, the computational cost associated with it is immense. Nevertheless, the accuracy of the model predictions supports the validity of this hypothesis, although it remains a conjecture due to the lack of definitive proof.

Finally, the models are used to predict the IC$_{50}$ of 50 \emph{azophotoswitches}, for which data on the free binding energy have been provided by the Physical Chemistry Unit at UAB \cite{UAB_ComputationalChemistry}. The statistical analysis of the predictions shows a clear correlation between both quantities, which supports the project's hypothesis.

\newpage
\thispagestyle{sumari}
%\rhead{\TitolTFG}
\tableofcontents{}

\newpage
\rhead{\TitolTFG}
\section{List of abbreviations}
\renewcommand{\glsnamefont}[1]{\textbf{#1}}
\printnoidxglossary[type=main, title={\vspace{-1cm}}, nonumberlist, nogroupskip, style=super]%nonumberlist deletes the list of links in the index to every acronym listed. nogroupskip deletes the extra space between letter groups
\newpage
\listoffigures{}
\listoftables{}
\newpage

\setcounter{page}{1}
\pagenumbering{arabic}
\section{Introduction}
The impact of \gls{ai} on science has been nothing short of a groundbreaking revolution, with few comparable precedents. The rapid advancements in \gls{ai} have transformed numerous scientific fields\cite{AlphaFold2BiologyAndMedicine}\cite{NationalLibraryOfMedicineGeneralArticle}, including computational chemistry. Today, one of the primary goals of computational chemistry is to predict the properties of unstudied substances while minimizing experimental costs. Traditional approaches in chemistry often rely on complex laboratory techniques, which, while effective, can be time-consuming, expensive, and resource-intensive. In contrast, computational chemistry offers a wide range of methods capable of predicting a molecule’s properties with reasonable accuracy. However, when \gls{ai} comes into play, predictions have demonstrated an almost surgical precision.

Perhaps one of the most representative events showcasing the enormous impact of \gls{ai} on chemistry is the 2024 Nobel Prize in Chemistry. The winners, David Baker\cite{NobelPrizeBale}, along with Demis Hassabis and John Jumper\cite{NobelPrizeJumper}, were not traditionally trained chemists. Instead, their expertise lies in \gls{ai} algorithms and \gls{ml} methods applied to protein research. This milestone, among others, triggered a surge of chemistry researchers diving into the world of \gls{ai}, seeking applications for their respective fields. Today, the thrilling progress in computational chemistry has been further reinforced by these cutting-edge tools\cite{MachineLearningPaper5Lipoxygenase}, and the rapid pace of development keeps the scientific community eagerly anticipating future applications in fields such as medicine, materials science, and beyond. In this project we aim to apply the new AI and ML algorithms to our object of study, the \gls{ptgs2} also known as \gls{cox2}, depicted in Figure (\ref{FigureCox2ChimeraX}), a protein tightly linked to the onset of numerous cancers forms\cite{Cox2CancerReview}. 

Although significant advancements have been made, cancer still accounts for over 8 million deaths per year worldwide, and the scientific and medical communities remain far from achieving its complete eradication. Inflammation is one of the hallmarks of carcinogenesis, in fact, various cancer therapies target inflammation as a means of preventing and reducing cancer occurrences. When a tissue is damaged, inflammation protects the organism from infections caused by external pathogens, a key function of the immune system to prevent the presence of invaders in the body. During the inflammatory process, cells proliferate under the command of the immune system to replace the damaged cells of the affected tissue. However, if this cell reproduction continues beyond the healing of the damaged tissue, it can potentially lead to cancer, contradicting the initial healing purpose of the inflammatory process. In some cases, inflammation can become chronic, leading to tumor development and uncontrolled cell proliferation. As a result, a wide range of drug prototypes have been designed to suppress inflammation. However, many of these drugs have been linked to severe side effects, including immunosuppression, cardiovascular risks, and gastrointestinal complications. Consequently, the administration of these drugs is often contrindicative, and the search for a more effective and safer treatment remains ongoing. Plently of the research in this field is mainly focused on the pro-inflamatory enzyme \gls{cox2}, one of the main commanders in the inflammation process and responsable to convert the arachidonic acid to prostaglandin \ch{H_2} (Scheme (\ref{Cox2MainReaction})). Subsequently, a therapy based on the chemical inhibition of the \gls{cox2} with no side effects has been one of the research lines in cancer treatment leading to a considerable amount of experiments and data.
\begin{figure}[t]
	\centering
	\includegraphics[width = \textwidth, trim={10cm 0cm 10cm 0cm}, clip]{GeneralSources/Cox2ChimeraX.jpg}
	\caption{Structural representation of the \gls{cox2} generated by the ChimeraX software\cite{ChimeraX}.}
	\label{FigureCox2ChimeraX}
\end{figure}

The increasing interest in \gls{cox2} inhibitors has granted the scientific community with an extensive database of molecular inhibition potentials for this protein. In this project, we focus on the Half Maximal Inhibitory Concentration (\gls{ic50}), a standard metric representing the concentration of a drug required to inhibit 50\% of a target protein’s activity. Related measures include \gls{ic90} and \gls{ic99}, which correspond to 90\% and 99\% inhibition, respectively. The lower the \gls{ic50} value of a molecule, the lower the concentration needed to inhibit \gls{cox2}, indicating higher efficiency. This factor is crucial in drug design, as a lower required dosage minimizes the presence of foreign substances in the body, thereby reducing the risk of adverse effects\footnote{Naturally, multiple other factors influence drug side effects.}.

\begin{figure}[H]
\captionsetup{type=scheme}
\centering
\schemestart
\scalebox{0.7}{
\chemname{\chemfig[angle increment=30, scale=0.1]{
HO-[1](=[3]O)-[-1]-[1]-[-1]-[1]=[-1]-[-3]-[-1]=[-3]-[7]-[-3]=[7]-[-7]-[7]=[-7]-[3]-[-7]-[7]-[-7]-[7]
}}{Arachidonic acid}
}
	\arrow{->[\gls{cox2}]}
\scalebox{0.7}{
\chemname{\chemfig[angle increment=30]{OH-[5](=[3]O)-[7]-[-3]-[7]-[-3]=[7]-[5]>:[7]*6(-(<[-3.8])-O-O-(<[3.8]C)-(<[-1]=[1]-[-1](<:[-3]OH)-[1]-[-1]-[1]-[-1]-[1])-)
}}{Prostaglandin \chemfig{H_2}}
}
\schemestop
\caption{Reaction catalysed by the COX-2 between arachidonic acid to prostaglandin H$_2$.}
\label{Cox2MainReaction}
\end{figure}

Determining an experimental \gls{ic50} value can be both costly and time-consuming\footnote{Usually this parameter is computed throughout the Cheng Prusoff Equation (Def. (\ref{definitionChengPrusoff})) with experimental data.}. On the other hand, \gls{ai} provides an alternative by offering highly accurate predictions based on existing data, optimising research processes, and accelerating scientific discovery. In this scenario this project aims to implement artificial intelligence in computational chemistry, concretely, using \gls{ai}-based algorithms to predict a drug’s inhibition potential\cite{BookIC50} for a given protein with a relativelly good accuracy \footnote{The accuracy of the ML models is discussed on the Results and Discussion section (\ref{sectionResultsAndDiscussion}).}. To achieve this, we make use of the ChEMBL database\cite{ChemblDatabase}, a vast repository of bioactive molecules with drug-like properties. We extract all known molecular data with a documented inhibition potential for the target protein, creating a comprehensive dataset. The chemical descriptors of each molecule in the database are then computed using AlvaDesk\cite{MauriMolecularDescriptorsBook}\cite{AlvaDescSecondPaper} software. Around 5000 descriptors are calculated\cite{DescriptorsBook}, which comprehend from the elemental molecular weight to the complex equipotential electronic surface, providing critical information about each compound’s behaviour. The resulting dataset is subsequently used to train \gls{ai} models, enabling them to predict the inhibition potential of unknown compounds. Finally, we evaluate the reliability of each model by testing it against real experimental data. 

It is important to emphasise the central hypothesis of this project: \emph{There exists a combination (or combinations) of chemical descriptors that are directly correlated with the inhibition of the protein}. While this idea may seem fundamental, it remains unproven due to the complexity of molecular interactions and the vast number of possible descriptor combinations. Despite significant progress in computational chemistry, identifying the exact descriptors that govern inhibition potential has been a persistent challenge. The lack of an ultimate proof underscores the need for advanced computational techniques. By analysing large datasets, \gls{ai} can detect hidden correlations that may not be immediately apparent through traditional statistical methods.

The \gls{ai} algorithm used is a in this study is a \gls{ml} model known as the \gls{rf} algorithm\cite{MachineLearningRandomForest}\cite{UAB_ComputationalChemistry}, a powerful ensemble learning method that generates multiple decision trees and combines their outputs to improve prediction accuracy. This approach is particularly well-suited for computational chemistry due to its ability to handle large datasets, manage complex relationships between variables, and reduce overfitting. The Random Forest algorithm operates by constructing numerous random decision trees, each trained on different subsets of the dataset. The final prediction is obtained by averaging the outputs of all trees, ensuring robust and reliable results.

Moreover, the choice of the Random Forest (\gls{rf}) algorithm is motivated by the presence of decision trees in various chemistry-related fields. In spectroscopy, for instance, decision trees are used in group theory to classify molecular symmetry. Similarly, in analytical chemistry, decision trees assist in substance separation techniques, while in organic chemistry, they are used to model reaction pathways.

By applying \gls{ai} models to \gls{cox2}, we assess their compatibility with the latest research findings\cite{Cox2InhibitorsReview}, demonstrating \gls{ai}’s potential as a powerful tool in computational chemistry research. Our approach not only validates \gls{ai}’s effectiveness in predicting inhibition potential but also provides insights into the underlying molecular mechanisms governing \gls{cox2} interactions. Additionaly, this study aims to bridge the gap between \gls{ai} and computational chemistry, reinforcing the \gls{ai}’s potential to revolutionise drug discovery and molecular research. The ability to predict inhibition potential with high accuracy can accelerate the development of new pharmaceuticals, reduce reliance on costly laboratory experiments, and contribute to a more efficient drug screening process. Furthermore, identifying key molecular descriptors correlated with inhibition could lead to a deeper understanding of chemical interactions, opening new avenues for research in medicinal chemistry and bioinformatics.

Currently, the pharmacological therapies for the \gls{cox2} inhibition are based on \gls{nsaid}\cite{nsaidDefinition} such as the popular ibuprofen or aspirin. However they have proved to be related to cardiovascular diseases and are contraindicated for people with more than 50 years or people with gastrointestinal problems, among others. Still some alternative therapies related with \gls{nsaid}s are being explored, in particular, one of the most revolutionary ideas is the application of azophotoswitches in drug design.

We define a molecule as an azophotoswitch if it contains an azo bond (\ch{N=N}) that is sensitive to configurational transformations upon photo-excitation. For instance, let us consider the case of (E/Z)-N,1-diphenylmethanimine,
\begin{figure}[H]
\captionsetup{type=scheme}
\centering
\schemestart
\scalebox{0.8}{
	\chemfig[angle increment=30]{*6(-=-(-[1]N=[-1]N-*6(-=-=-=))=-=)}}
	\arrow{<=>[$h\nu$]}
\scalebox{0.8}{
	\chemfig[angle increment=30]{
	N(-[-4]*6(-=-=-=))=[0]N-[-2]*6(-=-=-=)
	}}
\schemestop
	\caption{(E/Z)-N,1-diphenylmethanimine conversion as an example of an azophotoswitch.}
\label{AzophotoswitchExampleFigure}
\end{figure}
The photo-induced E/Z isomerization leads to distinct bioactivities for each configuration. This variation arises from the stereochemical constraints required for a molecule to bind to a target protein. Typically, a protein’s binding site has a specific shape, and only molecules whose conformation matches this shape can interact effectively. In the context of an azophotoswitch, one isomer may fit precisely into the binding pocket, leading to strong interactions, while the other may not. Accordingly, we refer to the isomer that interacts most effectively with the protein as the active configuration, and the other as the inactive configuration.\footnote{In the context of this project, interaction refers to inhibition of the protein’s activity.} \footnote{Generally, the trans-isomer is considered the active configuration, while the cis-isomer is considered inactive. However, exceptions exist.}

The primary application of azophotoswitches in drug design is the administration of the inactive isomer, which is assumed to be non-toxic to the organism. Later, the active configuration is generated through selective photo-excitation at the target site. This strategy minimizes drug activity in unintended tissues, thereby reducing side effects. As a result, higher dosages may be administered safely by localizing the therapeutic effect to the desired area. These therapies are still in the experimental stage and remain primarily within research and development.

Among the most promising drug candidates for \gls{cox2} inhibition are Celecoxib (Definition \ref{definitionCelecoxib}) and Rofecoxib (Definition \ref{definitionRofecoxib}).\footnote{These compounds are often referred to as coxib drugs due to the common suffix -coxib.} At the Physical Chemistry Unit of the UAB \cite{UAB_ComputationalChemistry}, researchers are investigating azophotoswitches as potential COX-2 inhibitors, using molecular structures inspired by Celecoxib. Various computational analyses have been conducted, with particular attention to the binding free energy ($\Delta G_{\text{binding}}$).\footnote{Additional results are provided in Appendix \ref{appendixTableOfResults}.}

As an application of our \gls{ml}-based predictive models, we aim to estimate the \gls{ic50} values of azophotoswitch prototypes.\footnote{These values are predictions, as no experimental or simulated data are currently available.} We will then compare these predictions with the computed $\Delta G_{\text{binding}}$ values, as both metrics are indicative of inhibition efficacy. While a direct linear correlation is not expected, we hypothesize that lower \gls{ic50} values should correspond to lower (more negative) $\Delta G_{\text{binding}}$ values, reflecting stronger binding affinities. These relationships will be explored in detail in Section \ref{sectionResultsAndDiscussion}.

In conclusion, this study seeks to bridge the gap between Artificial Intelligence and computational chemistry, demonstrating AI’s potential to revolutionize drug discovery and molecular research. Accurate predictions of inhibition potential could accelerate the development of new pharmaceuticals, reduce the need for costly laboratory experiments, and streamline the drug screening process. Furthermore, identifying key molecular descriptors linked to inhibition may provide valuable insights into chemical interactions, opening new directions for research in medicinal chemistry and bioinformatics.
\newpage
\section{Objectives}



\newpage
\section{Methodology}\label{sectionMethodology}
The source code is all stored in the \href{https://github.com/SirSergi0/Repository---AI-application-for-azophotoswitches-optimization-with-pharmacological-interest}{\emph{AI application for azophotoswitches optimization with pharmacological interest} GitHub repository}\cite{GitHub}.

The target protein's ID is set at \emph{CHEMBL230} corresponding to the COX-2 ID in the ChEMBL database\cite{ChemblDatabase}. Utilising \emph{requests} python package\cite{PythonPackageRequests} a query URL is sent asking for all molecules with a know $IC_{50}$ value (Def. \ref{definitionIC50}) with a limit of 1000 entries per request. The process is iterated until all data is extracted leading a total of 7979 molecules. Hence the datasheet is processed in pandas dataframes\cite{PythonPackagePandas} and encrypted into binary feather files to optimise reading-writting speed. By removing entries with the same canonical smiles a total of 5112 molecules remain. Among this entries well known drugs such as Celecoxib (Def. \ref{definitionCelecoxib}), Rofecoxib (Def. \ref{definitionRofecoxib}) or even Ibuprofen can be found. However the $IC_{50}$ molecules range is comprehended from $10^{-3}$ to $10^{8}$ nM, a counterproductive range for the AI training procedure. Since we are interested in testing azophotoswitches with presumably low \gls{ic50}, training \gls{ml} models with data of the order of $10^{8}$ or $10^{3}$ nM can be counterproductive since the model might misinterpret the data.\footnote{Typically the range of the azophotoswitches \gls{ic50} is arround the Celecoxib's \gls{ic50}, i.e. arround $120$ nM}. Consecutivelly, a hard-coded range is filtered discarding all molecules outside the given range, for the most part of the analysis this range is set at $[0,200]$ nM \footnote{this $IC_{50}$ working range is the standard in this kind of studies \cite{MachineLearningPaper5Lipoxygenase}.} which reduces the dataset to 1438 entries (i.e. molecules).

With the AlvaDesk-python \cite{AlvaDescSecondPaper} facility, the chemical descriptors (i.e., the chemical fingerprint) of each molecule are computed, providing a total of $5800$ descriptors per molecule. Still, some chemical descriptor need from the presence of a certain atom, for instance a chemical descriptor related to presence of a metal in the molecule, or active gorup, for example chemical descriptor related to the presence of a carboxilic acid, or chemical descriptor related triple bound among other possiblities. The computation of this chemical descriptors appear as \emph{null values}. By deleting molecular descriptors with null values 2917 chemical descriptors remain. Here, the Pearson correlation coefficient (Def. (\ref{definitionPersonCoefficient})) between each chemical descriptor and the $IC_{50}$ value is computed, providing insight into the direct relationship between $IC_{50}$ and the descriptors. This relationship will be discussed in the Results and Discussion Section (\ref{sectionResultsAndDiscussion}).

At this stage, the average $IC_{50}$ is calculated, and the neighborhood size corresponding to the percentage defined by the hard-coded variable \emph{percentageErased} is removed. This allows us to distinguish between \emph{highly active molecules} and \emph{least active molecules}, those with lower and higher $IC_{50}$ values, respectively. By doing so, it is possible to compute the classification accuracy statistics of the model from the cluster association of each prediction. Thus we can compare each prediction of each molecules to experimental data, i.e. the model predicts a molecule to be highly active molecule to a highly active molecule this will be denoted as a \gls{tp}. Similarly we can define a \gls{tn}, \gls{fp} and \gls{fn} . This procedure is an standard for evaluating a \gls{ml} model realiability. 

Subsequently, each set of substances is randomly divided into two datasets: a \emph{training set} for training the \gls{rf} algorithm and a \emph{testing set} for testing the statistics and realiability of the \gls{rf} model, following the proportion specified by the hard-coded variable \emph{testSizeProportion}. This procedure is illustrated in Figure (\ref{FigureDataSplittingDiagram}).
\begin{figure}[H]
\centering
\includegraphics[width = \textwidth]{GeneralSources/DataSplittingDiagram.pdf}
\caption{Splitting and processing data's scheme.}
\label{FigureDataSplittingDiagram}
\end{figure}
Afterwards, a Random Forest (\gls{rf}) algorithm is trained using the Sticky Learn \cite{PythonPackageStickitLearn} Python package, which is supported by Microsoft and Google among others. Although the project’s primary focus is not the detailed workings of Machine Learning (\gls{ml}) algorithms but its applications in computational chemestry, it is natural to ponder the essence of what the machine is doing. The name “Random Forest” is directly tied to the analogy between a tree and a decision tree. \par

As described in Definition (\ref{definitionDecisionTree}), the Random Forest algorithm generates a collection of random decision trees. The number of trees in the forest is a configurable parameter of the algorithm. More specifically, the Random Forest algorithm produces a set of decision trees, each trained on a random subset of the training data. The structure of each tree adjusts during training to make predictions that closely approximate the true target values. The final output of the Random Forest is typically an aggregate of the predictions from all the individual trees, such as a majority vote for classification tasks or an average for regression tasks. \par

One notable characteristic of this method is its inherent randomness. Each Random Forest model is unique, even if configured with the same parameters and initialized under identical conditions. However, experimental evidence suggests that this randomness does not significantly impact the results. If two Random Forest algorithms are trained on the same dataset using identical parameters, their outputs will converge, even if the internal structures of their decision trees differ.

The \gls{rf} algorithm generates a large number of random decision trees (determined by the hard-coded variable \emph{numberOfTrees}), which are trained with the training sets. Typically the more trees the model is trained with the best accuracy in the predictions, nevertheless the computational cost for each model training also increases with the number of trees. It is key to remark the predictions' accuracy can not be arbitrarily improved by only increasing the number of trees of the \gls{rf} model. The predictions' accuracy is bonded to the data the model is trained with, thus if the data does not provide enough information about the insight of the protein, more accurate predictions can not be archived independently of the number of trees of the model. As a general criteria we can affirm that,
\begin{align*}
	\uparrow\text{Data}\quad+\uparrow\text{Trees}\implies\uparrow\text{Predictions' accuracy}\\
	\downarrow\text{Data}\quad+\downarrow\text{Trees}\implies\downarrow\text{Predictions' accuracy}
\end{align*}
but more data or more trees independently will not lead to better accuracy. Then, for the data we dispose, predictions with 500 trees are the same as the ones with $300$ trees.

Later, these models are evaluated by predicting the $IC_{50}$ values of the \emph{testing sets}. Using the results, the \emph{True Positive Rate} (Def. \ref{definitionTruePositiveRate}), \emph{True Negative Rate} (Def. \ref{definitionTrueNegativeRate}), \emph{Classification Accuracy} (Def. \ref{definitionClassificationAccuracy}), and \emph{Matthews Correlation Coefficient} (Def. \ref{definitionMatthewsCorrelationCoeficient}) are computed. Based on these computational results, the variables \emph{percentageErased}, \emph{testSizeProportion}, and \emph{numberOfTrees} are manually adjusted to obtain the best results.

Finally, the \gls{rf} model with best statistics\footnote{The ruling criteria used to classify one model to have "better statistics" than other is discussed in section (\ref{sectionResultsAndDiscussion})} are used to predict the \gls{ic50} of the azophotoswitch prototipes classified in Appendix (\ref{appendixAzophotoshiches}). This procedure begins with the computation of the chemical descriptors with AlvaDesk software \cite{MauriMolecularDescriptorsBook}, then the pertinent chemical descriptors are droped and remaining only the ones the algorithm is trained with. Then, by inputing the computed descriptors to the \gls{rf} model the predicted \gls{ic50} pop out as an outcome and are analyzed.

\newpage
\section{Results and Discussion}\label{sectionResultsAndDiscussion}

\subsection{Available data}
In Figure (\ref{figureGraphicRecurrecyIC50}) we present the extracted \gls{ic50} data extracted from the ChEMBL database under the target ID \emph{CHEMBL230} for the interval $[0,200]$ nM. Figure (\ref{figureGraphicRecurrecyIC50}) clearly shows multiples of 10 are significantly overrepresented respect to other values. Additionally, by analyzing data from other proteins we can confirm this is not an exception but rather the norm. It looks like experimental chemists have a preference for multiples of 10, often rounding their results to the nearest value in this category. Consequently, an inherent error associated with these values will always be present, and we will have to deal with it. This observation will be one of the cornerstones of our ultimate results.

\begin{figure}[H]
	\centering
	\includegraphics[width = \textwidth, trim={0.85cm 0.3cm 1.5cm 1.35cm}, clip]{../Plots/DataGapRepresentationIC50percentageEresed15.0Gap_min_32Gap_max_53.pdf}
	\caption{\gls{ic50} values from our dataset and generation of a 20\% gap.}
	\label{figureGraphicRecurrecyIC50}
\end{figure}

Additionally in Figure (\ref{figureGraphicRecurrecyIC50}) we can also observe the procedure of the generation of a gap in the data. Giving a set of data the average \gls{ic50} is computed, for this case it is arround $47$ nM. Then, a percentage of the data is erased ruled by the \emph{percentageErased} hard-coded variable in a way that the average \gls{ic50} is located in the middle of the interval. In Figure (\ref{figureGraphicRecurrecyIC50}) all the data is presented in color blue + orange, then the gap is performed by the delation of the blue region centered around $47$ nM. For this case this gap comprehend the 20\% of the data.

\subsection{Correlation of individual descriptors and \gls{ic50}}

Afterwards the chemical descriptors are computed with AlvaDesk software. Hence, we proceed to calculate the Pearson Correlation Coefficients\footnote{The Pearson Correlation Coefficient is also denoted as $r^2$} as defined in Definition (\ref{definitionPersonCoefficient}) between the descriptors values and the \gls{ic50} of each molecule. The results are subsequently plotted in Figure (\ref{figureGraphicCorrelationFactor}). From the plot, we observe a‚ Gaussian bell shape centered around zero, which aligns with our expectations. This suggests that there is no inherently preferred set of chemical descriptors for the protein, a finding that may initially seem at odds with the project's goals.

\begin{figure}[H]
	\centering
	\includegraphics[width = \textwidth, trim={0.6cm 0.3cm 1.5cm 1.35cm}, clip]{../Plots/ChEMBL_ExtractorData_CHEMBL230_IC50_1000CorrelationFactors.pdf}
	\caption{Pearson correlation coefficient for each chemical descriptor.}
	\label{figureGraphicCorrelationFactor}
\end{figure}

This results is entirely natural, if a preferred descriptors would exist the implementation of \gls{ai}s would not be necessary and major advancedments would already be archived. Still our hypothesis holds, since it states that \emph{there exists a combination (or combinations) of chemical descriptors that are directly correlated with the inhibition of the protein}. This observation does not contradict that statement. Intuitively, we may believe that filtering out descriptors with the lowest  $r^2$  values might lead to improved results. This possibility will be explored in subsequent discussions.


\subsection{Pearson correlation factor and mean squared error}

Once the models are generated, i.e. the pertinent variables are set up, predictions can been. Thus, we can compute the Pearson Correlation Factors (Def. (\ref{definitionPersonCoefficient})) and Mean Squared Error (Def. (\ref{definitionMeanSquaredError})) for the predictions versus the experimental \gls{ic50} data. In Figure (\ref{FigPlotErasedPercentageVSRSquare}) we present two plots of the Pearson Correlation Factors and Mean Squared Error in terms of the erased percentage.

\begin{figure}[H]
\begin{tikzpicture}
\begin{axis}[
    width=0.45\textwidth,
    height=8cm,
    xlabel={$\%_{\text{Erased}}$},
    ylabel={$r^2$},
    xmin=0, xmax=100,
    ymin=0.94, ymax=1.0,
    xtick={0, 10, 20, 30, 40, 50, 60, 70, 80, 90, 100},
    ytick={0.9, 0.92, 0.94, 0.96, 0.98, 1.0},
    legend pos=south east,
    legend style={font=\small},
    tick label style={font=\small},
    label style={font=\small},
]
\addplot[
    color=magenta,
    mark=diamond*,
    thick
] table[x index=0, y index=1, col sep=space] {GeneralSources/ErasedPercentageVSRSquare.txt};
\end{axis}
\end{tikzpicture}
\begin{tikzpicture}
\begin{axis}[
    width=0.45\textwidth,
    height=8cm,
    xlabel={$\%_{\text{Erased}}$},
	ylabel={Mean Squared Error (nM$^2$)},
    xmin=0, xmax=100,
    ymin=0, ymax=200,
    xtick={0, 10, 20, 30, 40, 50, 60, 70, 80, 90, 100},
    legend pos=north east,
    legend style={font=\small},
    tick label style={font=\small},
    label style={font=\small}
]
\addplot[
    color=magenta,
    mark=diamond*,
    thick
] table[x index=0, y index=1, col sep=space] {GeneralSources/ErasedPercentageVSMeanSquaredError.txt};
\end{axis}
\end{tikzpicture}
\caption{Plots of $\%_{\text{Erased}}$ vs $r^2$ and $\%_{\text{Erased}}$ vs Mean Squared Error. For a maximum $IC_{50}$ = 200 nM and a minimum of 0 nM,  taking into account all the chemical descriptors, the 20.0\% of the remaining data is saved for testing and the  80.0\% for training. The training is done with 200 trees.}
\label{FigPlotErasedPercentageVSRSquare}
\end{figure}

From Figure (\ref{figureGraphicRecurrecyIC50}), we know the data has an inherent precision problem, especially when approaching higher  $IC_{50}$  values, additionally we do not now the precision of some measurements. Since our working models do not account for the lack of precision in each molecule's data, the values of  $r^2$  and mean squared error are not expected to be exactly 1 and 0, respectively. In fact, some variance and uncertainty are inevitable. Therefore, discarding more than 50\% of our data would be a critical mistake, even if statistical measures suggest otherwise. However, these values still serve as a good reflection of our method's reliability up to a certain threshold.

In this work, we will arbitrarily set this threshold at  $r^2 = 0.90$ . This hard-coded decision is purely conventional, as we need to establish a point beyond which  $r^2$  values become meaningless (i.e., obtaining  $r^2$  values above 0.90 is unnecessary, given that such variance is inherent to our data). The justification for this choice is that, in some areas of chemistry\footnote{Particularly in analytical chemistry} two data sets are generally considered correlated if their  $r^2$  value exceeds 0.90. However, it is important to emphasize that this convention is purely subjective and should not be regarded as an inviolable rule. The reliable statistics will come from the \emph{True Positive Rate} (Def. \ref{definitionTruePositiveRate}), \emph{True Negative Rate} (Def. \ref{definitionTrueNegativeRate}), \emph{Classification Accuracy} (Def. \ref{definitionClassificationAccuracy}), and \emph{Matthews Correlation Coefficient} (Def. \ref{definitionMatthewsCorrelationCoeficient}), the standard quantities for rating a \gls{ml} model.

\subsection{Azophotoswitches analysis}

In Figure (\ref{figureGraphicGbindingIC50}) we present the results for the predicted \gls{ic50} in terms of the $\Delta G_{binding}$ provided by the Physical Chemistry Unit of the UAB. The exact results are stored in Appendix \ref{appendixTableOfResults} in Tables (\ref{TableMainResults1}) and (\ref{TableMainResults2}), additionall the conditions under which the Random Forest (\gls{rf}) algorithm has been trained and the statistics analysis of its predictions is stored in Table (\ref{TableConditionsMainResults}).

\begin{figure}[H]
	\centering
	\includegraphics[width = 0.7\textwidth, trim={1.75cm 0.8cm 2.5cm 2cm}, clip]{../Plots/GbindingVSIC50.pdf}
	\caption{Graphic of the azophotoswitches from Appendix \ref{appendixAzophotoshiches} where the x-axis represents the $\Delta G_{binding}$ and the y-axis the computed \gls{ic50}.}
	\label{figureGraphicGbindingIC50}
\end{figure}

Furthermore, we may also show the statistics and reliability of this results in Table (\ref{TableStatisitcsMainResults}).

\begin{table}[H]
    \centering	
	\caption{Statistics for the computation of the results from tables (\ref{TableMainResults1}) and (\ref{TableMainResults2}).}
	\label{TableStatisitcsMainResults}
	\begin{tabular}{c|c}
		Mean Squared Error       & 1202.70\\\hline
		$r^2$ 	                 & 0.90\\\hline
		ClassificationAcuracy    & 0.94\\\hline
		MatthewsCorrelationFactor& 0.87
    \end{tabular}
\end{table}

In Figure (\ref{figureGraphicGbindingIC50}) we can see a clear correlation between our initial intuition. Lower values of free eneregy (the most negative ones) correspond to low \gls{ic50} values i.e. highly active molecules

\subsection{Azophotoswitches analysis}

\begin{figure}[H]
	\centering
	\includegraphics[width = \textwidth, trim={0cm 0cm 0cm 0cm}, clip]{GeneralSources/LastFrameDockingTwoRingsFiveMembers.jpg}
	\caption{Structural ilustration of the molecule identied as \ref{figureCelecoxibTwoRings}.7 and \gls{cox2} docking generated by the ChimeraX software[15].}
	\label{figureDockingBestResult}
\end{figure}



\newpage
\section{Conclusions}

\newpage
\section{Acknowledgements}
I am grateful to Dr. Josep Maria Lluch, Dr. Miquel Moreno Ferrer and Dra. Àngels González Lafont for allowing me to be one of his pupils and for presenting this project. I am also deeply grateful to PhD Álex Pérez Sánchez and PhD Pedro Martínez Zaragoza for sharing their ideas and fruitful conversations.

I would like to acknowledge my mother and sister for their unconditional support, and my father for the values he taught me. Additionally, I would like to thank my friends Francisco Montaño, David Muñoz and Manel Martin for listening to my monologues about my excitement for the \gls{cox2} and azophotoswitches.

Molecular graphics and analyses performed with UCSF ChimeraX, developed by the Resource for Biocomputing, Visualization, and Informatics at the University of California, San Francisco, with support from National Institutes of Health R01-GM129325 and the Office of Cyber Infrastructure and Computational Biology, National Institute of Allergy and Infectious Diseases.

\newpage
\section{Bibliography}
\printbibliography[title = { \vspace{-1cm}}]
\newpage

\appendix
\section{Rellevant definitions}
\begin{definition}\label{definitionIC50}
$IC_{50}$: Half maximal inhibitory concentration assigned to the drug concentration required for a $50\%$ inhibition a protein. Other quantities such as $IC_{90}$ or $IC_{99}$ are also commonly used. However, $IC_{90}$ is generally approximated as 10 times the $IC_{50}$ concentration in virtue of experimental observations\cite{BookIC50}. For this project, we aim to identify substances with the lowest possible $IC_{50}$, as our goal is to minimize the presence of foreign substances in the living organism.
\end{definition}

\begin{definition}\label{definitionCelecoxib}
Celecoxib: \footnote{UPAC name: 4-[5-(4-Methylphenyl)-3-(trifluoromethyl)pyrazol-1-yl]benzenesulfonamide} drug known to be a selective COX-2 inhibitor (currently is not \emph{highly selective} respect to newer drugs), see Scheme (\ref{CelecoxibFigure}). It $IC_{50}$ value is $120$ nM.
\end{definition}

\begin{definition}\label{definitionRofecoxib}
Rofecoxib: \footnote{UPAC name: 4-(4-methylsulfonylphenyl)-3-phenyl-5H-furan-2-one} drug known to be a selective COX-2 inhibitor, see Scheme (\ref{RofecoxibFigure}). It $IC_{50}$ value is $180$ nM.
\end{definition}

\begin{figure}[H]
\captionsetup{type=scheme}
\centering
\scalebox{0.8}{
\chemfig[angle increment=30]{
H_2N-[1]S(=[4]O)(=[-2]O)-[1]*6(-=-(-[1]N*5(-N=(-[-1](-[-4]F)(-[-2]F)-[1])-=(-[4.5]*6(-=-(-[4.5])=-=))-))=-=)
}}
\caption{Chemical graph of Celecoxib.}
\addcontentsline{lof}{figure}{Chemical graph of Celecoxib.}
\label{CelecoxibFigure}
\end{figure}

\begin{figure}[H]
\captionsetup{type=scheme}
\centering
\scalebox{0.8}{
\chemfig[angle increment=30]{
-[-1]S(=[-4]O)(=[-2]O)-[1]*6(-=-(-[1]*5(--O-(=[2.25]O)-(-[4.5]*6(-=-=-=))=))=-=)
}}
\caption{Chemical graph of Rofecoxib.}
\addcontentsline{lof}{figure}{Chemical graph of Rofecoxib.}
\label{RofecoxibFigure}
\end{figure}
\begin{definition}\label{definitionChengPrusoff}
Cheng Prusoff equation: standard equation used for the experimental computation of the IC50.
	\begin{align*}
		K_i=\frac{IC_{50}}{1+\frac{[S]}{K_m}}
	\end{align*}
\end{definition}
where $K_i$ is the binding affinity, $[S]$ is the substrate concentration, $K_m$ is the Michaelis constant and \gls{ic50} the half maximal inhibitory concentration.

\begin{definition}\label{definitionPersonCoefficient}
Pearson correlation coefficient: Given set of pairs of data $\{(x_i,y_i)\}_{i=1}^n$ the pearson correlation factor $r_{xy}$ is defined as,
\begin{align}
r_{xy}=\frac{\sum_{i=1}^n(x_i-\bar{x})(y_i-\bar{y})}{\sqrt{\sum_{i=1}^n(x_i-\bar{x})^2}\sqrt{\sum_{i=1}^n(y_i-\bar{y})^2}},
\end{align}
where $\bar{x}$ and $\bar{y}$ stand for the average value of ${x_i}_{i=1}^n$ and ${y_i}_{i=1}^n$ respectively. Note that $r_{xy}\in[-1,1]$. Therefore the sign of $r_{xy}$ is tightly related to the sign of alinear regression, more precisely if $x>0$, "$y$" generally\footnote{We would like to remark that the word "generally" stands for "the majority of the cases", since "generally" is commonly interpreted as a non-scientific/non-objective word} increases when "$x$" increases, as well as if $x<0$, "$y$" decreases when "$x$" increases.
\end{definition}

\begin{definition}\label{definitionMeanSquaredError}
Mean Squared Error: Given set of pairs of data $\{(x_i,y_i)\}_{i=1}^n$ the Mean Squared Error is the quantity defined as,
\begin{align}
	MSE=\frac{1}{n}\sum_{i=1}^n(X_i-Y_i)^2
\end{align}
The name of this quantity is self-descriptive, since $X_i-Y_i$ is the error associated to the i-th prediction and the MSE is the mean of the squares of this erros.
\end{definition}

\begin{definition}\label{definitionDecisionTree}
A decision tree is a classification algorithm based on a series of ordered \emph{if} statements. The algorithm begins at the top of the tree, where a question is posed to the data. Depending on the answer, the data follows different branches, each corresponding to a subsequent question. This process is repeated at each node until the data reaches the bottom of the tree, where the path it has followed determines the classification of the given data.

This kind of algorithm is vividly present in the chemical landscape, for instance in the spectroscopy realm the determination of a molecule's symmetry group is provided by a decision tree such as Figure (\ref{figExScpectroDecisionTree}).
\begin{figure}[H]
\centering
\includegraphics[width = 0.5\textwidth]{GeneralSources/Point_group_determination_flowchart_v2.png}
\caption{Decision tree for determining the point group of a molecule}
\label{figExScpectroDecisionTree}
\end{figure}
\end{definition}


\begin{definition}\label{definitionTruePositiveRate}
True Positive Rate: quantity related to a Machine Learning Model's sensitivity defined as:
\begin{align}
\frac{TP}{TP+FN}
\end{align}
where $TP,FP,TN,FN$ stands for "True Positive", "False Positive", "True Negative" \& "False Negative" respectively.
\end{definition}

\begin{definition}\label{definitionTrueNegativeRate}
True Negative Rate: quantity related to a Machine Learning Model's specificity defined as:
\begin{align}
\frac{TN}{TN+FN}
\end{align}
where $TP,FP,TN,FN$ stands for "True Positive", "False Positive", "True Negative" \& "False Negative" respectively.
\end{definition}

\begin{definition}\label{definitionClassificationAccuracy}
Classification Accuracy: quantity related to a Machine Learning Model's efectiveness defined as:
\begin{align}
\frac{TP+TN}{TP+FP+TN+FN}
\end{align}
where $TP,FP,TN,FN$ stands for "True Positive", "False Positive", "True Negative" \& "False Negative" respectively.
\end{definition}

\begin{definition}\label{definitionMatthewsCorrelationCoeficient}
Matthews Correlation Coefficient: quantity related to a Machine Learning Model's prediction capacity defined as:
{\scriptsize
\begin{align}
\frac{(TP\times TN)-(FP\times FN)}{\sqrt{(TP+FP)\times(TP+FN)\times(TN+FP)\times(TN+FN)}}
\end{align}
}where $TP,FP,TN,FN$ stands for "True Positive", "False Positive", "True Negative" \& "False Negative" respectively.
A Matthews Correlation Coefficient equal to 1 stands for a perfect prediction a Matthews Correlation Coefficient equal to 0 indicates the predictions are no better than random guessing, and a Matthews Correlation Coefficient equal to -1 stand for a total disagreement between predictions and actual outcomes.
\end{definition}


\section{Tables of azophotoswitches}\label{appendixAzophotoshiches}
\begin{figure}[H]
\captionsetup{type=scheme}
\centering
\scalebox{0.8}{
\chemfig[angle increment=30]{
H_2N-S(=[3]O)(=[9]O)-*6(-=-(-N*5(-N=(-R_1)-(-[:45]N=[3]N-[:135]*6(-=-(-R_2)=(-R_3)-=))=-))=-=)
}}
\caption{Template for Celecoxib's azo-derivates with pyrazole as heterocycle.}
\label{figureCelecoxibPyrazole}
\end{figure}

\begin{table}[H]
\centering
\caption{Table of potential photoswitches derivated from Celecoxib's azo-derivates with pyrazole as heterocycle}
\label{tableCelecoxibPyrazole}
\begin{tabular}{c||c|c|c}
Identifier & \ch{R_1} & \ch{R_2} & \ch{R_3} \\\hline\hline
\ref{figureCelecoxibPyrazole}.1 & \ch{CF_3} & \ch{CH_2CH_3}  & \ch{H}\\\hline
\ref{figureCelecoxibPyrazole}.2 & \ch{CF_3} & \ch{CH_2CH_3}  & \ch{F}\\\hline
\ref{figureCelecoxibPyrazole}.3 & \ch{CF_3} & \ch{CH_3}  & \ch{F}\\\hline
\ref{figureCelecoxibPyrazole}.4 & \ch{CF_3} & \ch{OCH_3}  & \ch{H}\\\hline
\ref{figureCelecoxibPyrazole}.5 & \ch{CF_3} & \ch{OCH_3}  & \ch{F}\\\hline
\ref{figureCelecoxibPyrazole}.6 & \ch{CF_3} & \ch{CH_3}  & \ch{H}\\\hline
\ref{figureCelecoxibPyrazole}.7 & \ch{H} & \ch{CH_3}  & \ch{H}\\\hline
\ref{figureCelecoxibPyrazole}.8 & \ch{F} & \ch{CH_3}  & \ch{H}\\\hline
\ref{figureCelecoxibPyrazole}.9 & \ch{Cl} & \ch{CH_3}  & \ch{H}\\\hline
\ref{figureCelecoxibPyrazole}.10 & \ch{Br} & \ch{CH_3}  & \ch{H}\\\hline
\ref{figureCelecoxibPyrazole}.11 & \ch{CH_3} & \ch{CH_3}  & \ch{H}\\\hline
\ref{figureCelecoxibPyrazole}.12 & \ch{H} & \ch{CH_3}  & \ch{F}\\\hline
\ref{figureCelecoxibPyrazole}.13 & \ch{F} & \ch{CH_3}  & \ch{F}\\\hline
\ref{figureCelecoxibPyrazole}.14 & \ch{Cl} & \ch{CH_3}  & \ch{F}\\\hline
\ref{figureCelecoxibPyrazole}.15 & \ch{Br} & \ch{CH_3}  & \ch{F}\\\hline
\ref{figureCelecoxibPyrazole}.16 & \ch{CH_3} & \ch{CH_3}  & \ch{F}
\end{tabular}
\end{table}
\begin{figure}[H]
\captionsetup{type=scheme}
\centering
\scalebox{0.8}{
\chemname{\chemfig[angle increment=30]{
H_2N-S(=[3]O)(=[9]O)-*6(-=-(-N*5(-N=(-CF_3)-(-[:45]N=[3]N-[:135]*6(-=-(--[3])=N-=))=-))=-=)
}}{Pyridine derivative}}
\par
\vspace{0.5cm}
\scalebox{0.8}{
\chemname{\chemfig[angle increment=30]{
H_3C-S(=[3]O)(=[9]O)-*6(-=-(-N*5(-N=(-CF_3)-(-[:45]N=[3]N-[:135]*6(-=-(-)=N-=))=-))=-=)
}}{\ch{SO_2CH_3} group derivative}}
\caption{Scheme for Celecoxib azo-derivatives based on pyridine and \ch{SO_2CH_3} groups.}
\label{tableCelecoxibPyridine}
\end{figure}

\begin{figure}[H]
\captionsetup{type=scheme}
\centering
\scalebox{0.8}{
\chemfig[angle increment=30]{
H_2N-S(=[3]O)(=[9]O)-*6(-=-(-*5(-O-(-R_1)=(-[:45]N=[3]N-[:135]*6(-=-(-R_2)=(-)-=))-=))=-=)
}}
\caption{Template for Celecoxib azo-derivatives with furan as a heterocycle.}
\label{figureCelecoxibFuran}
\end{figure}

\begin{table}[H]
\centering
\caption{Table of potential photoswitches derivated from Celecoxib's azo-derivates with furan as heterocycle.}
\label{tableCelecoxibFuran}
\begin{tabular}{c||c|c}
Identifier & \ch{R_1} & \ch{R_2} \\\hline\hline
\ref{figureCelecoxibFuran}.1 & \ch{CF_3} & \ch{H} \\\hline
\ref{figureCelecoxibFuran}.2 & \ch{H} & \ch{H} \\\hline
\ref{figureCelecoxibFuran}.3 & \ch{F} & \ch{H} \\\hline
\ref{figureCelecoxibFuran}.4 & \ch{Cl} & \ch{H} \\\hline
\ref{figureCelecoxibFuran}.5 & \ch{Br} & \ch{H} \\\hline
\ref{figureCelecoxibFuran}.6 & \ch{CH_3} & \ch{H} \\\hline
\ref{figureCelecoxibFuran}.7 & \ch{CF_3} & \ch{F} \\\hline
\ref{figureCelecoxibFuran}.8 & \ch{H} & \ch{F} \\\hline
\ref{figureCelecoxibFuran}.9 & \ch{F} & \ch{F} \\\hline
\ref{figureCelecoxibFuran}.10 & \ch{Cl} & \ch{F} \\\hline
\ref{figureCelecoxibFuran}.11 & \ch{Br} & \ch{F} \\\hline
\ref{figureCelecoxibFuran}.12 & \ch{CH_3} & \ch{F}
\end{tabular}
\end{table}

\begin{figure}[H]
\captionsetup{type=scheme}
\centering
\scalebox{0.8}{
\chemfig[angle increment=30]{
H_2N-S(=[3]O)(=[9]O)-*6(-=-(-*5(-S-(-R_1)=(-[:45]N=[3]N-[:135]*6(-=-(-R_2)=(-)-=))-=))=-=)
}}
\caption{Template for Celecoxib azo-derivatives with thiophene as a heterocycle.}
\label{figureCelecoxibThiophene}
\end{figure}

\begin{table}[H]
\centering
\caption{Table of potential photoswitches derivated from Celecoxib's azo-derivates with thiophene as heterocycle.}
\label{tableCelecoxibThiophene}
\begin{tabular}{c||c|c}
Identifier & \ch{R_1} & \ch{R_2} \\\hline\hline
\ref{figureCelecoxibThiophene}.1 & \ch{F} & \ch{H} \\\hline
\ref{figureCelecoxibThiophene}.2 & \ch{H} & \ch{F} \\\hline
\ref{figureCelecoxibThiophene}.3 & \ch{Cl} & \ch{F} 
\end{tabular}
\end{table}


\begin{figure}[H]
\captionsetup{type=scheme}
\centering
\scalebox{0.8}{
\chemfig[angle increment=30]{
H_2N-S(=[3]O)(=[9]O)-*6(-=-(-*5(-\chembelow{N}{H}-(-R_1)=(-[:45]N=[3]N-[:135]*6(-=-(-R_2)=(-R_3)-=))-=))=-=)
}}
\caption{Template for Celecoxib azo-derivatives with pyrrole as a heterocycle.}
\label{figureCelecoxibPyrrole}
\end{figure}

\begin{table}[H]
\centering
\caption{Table of potential photoswitches derivated from Celecoxib's azo-derivates with pyrrole as heterocycle.}
\label{tableCelecoxibPyrrole}
\begin{tabular}{c||c|c|c}
Identifier & \ch{R_1} & \ch{R_2} & \ch{R_3} \\\hline\hline
\ref{figureCelecoxibPyrrole}.1 & \ch{CF_3} & \ch{CH_3} & \ch{H} \\\hline
\ref{figureCelecoxibPyrrole}.2 & \ch{Cl} & \ch{CH_3} & \ch{F} 
\end{tabular}
\end{table}

\begin{figure}[H]
\captionsetup{type=scheme}
\centering
\scalebox{0.8}{
\chemfig[angle increment=30]{
H_2N-S(=[3]O)(=[9]O)-*6(-=-(-*6(-=(-R_1)-(-N=[2]N-[4]*6(-=-(-R_2)=-=))=-=))=-=)
}}
\caption{Template for Celecoxib azo-derivatives with benzene in place of the original heterocycle.}
\label{figureCelecoxibBenzene}
\end{figure}

\begin{table}[H]
\centering
\caption{Table of potential photoswitches derivated from Celecoxib azo-derivatives with benzene in place of the original heterocycle.}
\label{tableCelecoxibBenzene}
\begin{tabular}{c||c|c}
Identifier & \ch{R_1} & \ch{R_2} \\\hline\hline
\ref{figureCelecoxibBenzene}.1 & \ch{CF_3} & \ch{CH_2CH_3} \\\hline
\ref{figureCelecoxibBenzene}.2 & \ch{CF_3} & \ch{NCH_3COCH_3} \\\hline
\ref{figureCelecoxibBenzene}.3 & \ch{CF_3} & \ch{NHCH_3} \\\hline
\ref{figureCelecoxibBenzene}.4 & \ch{CF_3} & \ch{OCH_3} \\\hline
\ref{figureCelecoxibBenzene}.5 & \ch{Cl} & \ch{CH_3} 
\end{tabular}
\end{table}

\begin{figure}[H]
\captionsetup{type=scheme}
\centering
\scalebox{0.8}{
\chemfig[angle increment=30]{
H_2N-S(=[3]O)(=[9]O)-*6(-=-(-*5(-\chembelow{N}{H}-*6(-=-(-)=(-[3]N=[4.5]N-[6]*6(-=-(-R_1)=-=))-=)--=))=-=)
}}
\caption{Template for Celecoxib azo-derivatives with indole ring as a heterocycle.}
\label{figureCelecoxibIndole}
\end{figure}


\begin{table}[H]
\centering
\caption{Table of potential photoswitches derivated from Celecoxib azo-derivatives with indole ring as a heterocycle.}
\label{tableCelecoxibIndole}
\begin{tabular}{c||c}
Identifier & \ch{R_1}  \\\hline\hline
\ref{figureCelecoxibIndole}.1 & \ch{H} \\\hline
\ref{figureCelecoxibIndole}.2 & \ch{F} 
\end{tabular}
\end{table}

\begin{figure}[H]
\captionsetup{type=scheme}
\centering
\scalebox{0.7}{
\chemfig[angle increment=30]{
H_2N-S(=[3]O)(=[9]O)-*6(-=-(-N=[-1.5]N-[-3]*5(=-\chembelow{N}{H}-*6(-=-(-)=(-*6(-=-(-)=-=))-=)--))=-=)
}}
\caption{Template for Celecoxib azo-derivatives with indole ring as a heterocycle.}
\end{figure}

\begin{figure}[H]
\captionsetup{type=scheme}
\centering
\scalebox{0.7}{
\chemfig[angle increment=30]{
H_2N-S(=[3]O)(=[9]O)-*6(-=-(-N=[-1.5]N-[-3]*5(=-R_1-*5(-R_2-(-)=(-*6(-=-(-R_3)=-=))-=)--))=-=)
}}
\caption{Template for Celecoxib azo-derivatives with two rings of five members joint as a heterocycle.}
\label{figureCelecoxibTwoRings}
\end{figure}

\begin{table}[H]
\centering
\caption{Table of potential photoswitches derivated from Celecoxib azo-derivatives with two rings of five members joint as a heterocycle.}
\label{tableCelecoxibTwoRings}
\begin{tabular}{c||c|c|c}
Identifier & \ch{R_1} & \ch{R_2} & \ch{R_3} \\\hline\hline
\ref{figureCelecoxibTwoRings}.1 & \ch{NH} & \ch{NH} & \ch{H} \\\hline
\ref{figureCelecoxibTwoRings}.2 & \ch{NH} & \ch{O} & \ch{H} \\\hline
\ref{figureCelecoxibTwoRings}.3 & \ch{O} & \ch{NH} & \ch{H} \\\hline
\ref{figureCelecoxibTwoRings}.4 & \ch{O} & \ch{O} & \ch{H} \\\hline
\ref{figureCelecoxibTwoRings}.5 & \ch{NH} & \ch{NH} & \ch{CH_3} \\\hline
\ref{figureCelecoxibTwoRings}.6 & \ch{NH} & \ch{O} & \ch{CH_3} \\\hline
\ref{figureCelecoxibTwoRings}.7 & \ch{O} & \ch{NH} & \ch{CH_3} \\\hline
\ref{figureCelecoxibTwoRings}.8 & \ch{O} & \ch{O} & \ch{CH_3}
\end{tabular}
\end{table}

\section{Table of results}\label{appendixTableOfResults}
\begin{table}[H]
    \centering
	\caption{Results for the $\Delta$G$_{binding}$ and \gls{ic50}. The conditions and estatistics under which this computations have been done are stored in Table (\ref{TableConditionsMainResults}).}
	\label{TableMainResults1}
	\begin{tabular}{c|c|c|c|c|c|c}
		Type & Identifier & $R_1$ & $R_2$ & $R_3$ & $\Delta$ G$_{binding}$ (Kcal/mol) & Predicted \gls{ic50} (nM) \\ \hline\hline
		Pyrazole & \ref{figureCelecoxibPyrazole}.1 & \ch{CF_3} & \ch{CH_2CH_3}& \ch{H} & -4.3505 & 145 \\ \hline
        Pyrazole & \ref{figureCelecoxibPyrazole}.2 & \ch{CF_3} & \ch{CH_2CH_3}& \ch{F} & -3.619 & 160 \\ \hline
        Pyrazole & \ref{figureCelecoxibPyrazole}.3 & \ch{CF_3} & \ch{CH_3} & \ch{F} & -0.5452 & 192 \\ \hline
        Pyrazole & \ref{figureCelecoxibPyrazole}.4 & \ch{CF_3} & O\ch{CH_3} & \ch{H} & 9.9793 & 306 \\ \hline
        Pyrazole & \ref{figureCelecoxibPyrazole}.5 & \ch{CF_3} & O\ch{CH_3} & \ch{F} & 1.5591 & 238 \\ \hline
        Pyrazole & \ref{figureCelecoxibPyrazole}.6 & \ch{CF_3} & \ch{CH_3} & \ch{H} & -5.3292 & 142 \\ \hline
        Pyrazole & \ref{figureCelecoxibPyrazole}.8 & \ch{F} & \ch{CH_3} & \ch{H} & 5.0694 & 240 \\ \hline
        Pyrazole & \ref{figureCelecoxibPyrazole}.9 & \ch{Cl} & \ch{CH_3} & \ch{H} & -1.1579 & 171 \\ \hline
        Pyrazole & \ref{figureCelecoxibPyrazole}.10 & \ch{Br} & \ch{CH_3} & \ch{H} & 6.3225 & 247 \\ \hline
        Pyrazole & \ref{figureCelecoxibPyrazole}.11 & \ch{CH_3} & \ch{CH_3} & \ch{H} & 3.0677 & 215 \\ \hline
        Pyrazole & \ref{figureCelecoxibPyrazole}.12 & \ch{H} & \ch{CH_3} & \ch{F} & 1.0334 & 196 \\ \hline
        Pyrazole & \ref{figureCelecoxibPyrazole}.13 & \ch{F} & \ch{CH_3} & \ch{F} & 3.2216 & 220 \\ \hline
        Pyrazole & \ref{figureCelecoxibPyrazole}.14 & \ch{Cl} & \ch{CH_3} & \ch{F} & 5.1623 & 231 \\ \hline
        Pyrazole & \ref{figureCelecoxibPyrazole}.15 & \ch{Br} & \ch{CH_3} & \ch{F} & 5.6163 & 235 \\ \hline
        Pyrazole & \ref{figureCelecoxibPyrazole}.16 & \ch{CH_3} & \ch{CH_3} & \ch{F} & -3.3079 & 149 \\ \hline
        Furan & \ref{figureCelecoxibFuran}.1 & \ch{CF_3} & \ch{H} &  & -0.9611 & 171 \\ \hline
        Furan & \ref{figureCelecoxibFuran}.2 & \ch{H} & \ch{H} &  & 0.193 & 189 \\ \hline
        Furan & \ref{figureCelecoxibFuran}.3 & \ch{F} & \ch{H} &  & -4.8383 & 133 \\ \hline
        Furan & \ref{figureCelecoxibFuran}.4 & \ch{Cl} & \ch{H} &  & -2.4267 & 158 \\ \hline
        Furan & \ref{figureCelecoxibFuran}.5 & \ch{Br} & \ch{H} &  & -3.4217 & 144 \\ \hline
        Furan & \ref{figureCelecoxibFuran}.6 & \ch{CH_3} & \ch{H} &  & -3.9711 & 139 \\ \hline
        Furan & \ref{figureCelecoxibFuran}.7 & \ch{CF_3} & \ch{F} &  & -5.4022 & 148 \\ \hline
        Furan & \ref{figureCelecoxibFuran}.8 & \ch{H} & \ch{F} &  & -6.55 & 122 \\ \hline
        Furan & \ref{figureCelecoxibFuran}.9 & \ch{F} & F &  & -0.1311 & 189\\ \hline
        Furan & \ref{figureCelecoxibFuran}.10 & \ch{Cl} & \ch{F} &  & -5.8185 & 127\\ \hline
        Furan & \ref{figureCelecoxibFuran}.11 & \ch{Br} & \ch{F} &  & 6.1474 & 241
            \end{tabular}
\end{table}

\begin{table}[H]
    \centering	
	\caption{Results for the $\Delta$G$_{binding}$ and \gls{ic50}. The conditions and estatistics under which this computations have been done are stored in Table (\ref{TableConditionsMainResults}).}
	\label{TableMainResults2}
	\begin{tabular}{c|c|c|c|c|c|c}
		Type & Identifier & $R_1$ & $R_2$ & $R_3$ & $\Delta$ G$_{binding}$ (Kcal/mol) & Predicted \gls{ic50} (nM) \\ \hline\hline
        Thiophene & \ref{figureCelecoxibThiophene}.1 & \ch{F} & \ch{H} &  & 1.2724 & 192\\ \hline
        Thiophene & \ref{figureCelecoxibThiophene}.2 & \ch{H} & \ch{F} &  & -2.7842 & 152\\ \hline
        Thiophene & \ref{figureCelecoxibThiophene}.3 & \ch{Cl} & \ch{F} &  & -5.7964 & 120\\ \hline
        Pyrrole & \ref{figureCelecoxibPyrrole}.1 & \ch{CF_3} & \ch{CH_3} & \ch{H} & -3.5279 & 151\\ \hline
        Pyrrole & \ref{figureCelecoxibPyrrole}.2 & \ch{Cl} & \ch{CH_3} & \ch{F} & -1.3499 & 164\\ \hline
        Benzene & \ref{figureCelecoxibBenzene}.1 & \ch{CF_3} & \ch{CH_2CH_3}&  & -1.339 & 168\\\hline
        Benzene & \ref{figureCelecoxibBenzene}.2 & \ch{CF_3} & \ch{NCH_3COCH_3} &  & -1.9922 & 208\\ \hline
        Benzene & \ref{figureCelecoxibBenzene}.3 & \ch{CF_3} & \ch{NHCH_3} &  & 4.8759 & 232\\ \hline
        Benzene & \ref{figureCelecoxibBenzene}.4 & \ch{CF_3} & \ch{OCH_3} &  & -3.4432 & 155\\ \hline
        Indole & \ref{figureCelecoxibIndole}.1 & \ch{H} &  &  & -9.7543 & 74\\ \hline
        Indole & \ref{figureCelecoxibIndole}.2 & \ch{F} &  &  & -6.3397 & 108\\ \hline
        Indole & \ref{figureCelecoxibIndole}.3 &  &  &  & -5.532 & 111\\ \hline
        TwoRings & \ref{figureCelecoxibTwoRings}.2 & \ch{NH} & O & \ch{H} & -2.2146 & 165\\ \hline
        TwoRings & \ref{figureCelecoxibTwoRings}.3 & O & \ch{NH} & \ch{H} & -0.7334 & 178\\ \hline
        TwoRings & \ref{figureCelecoxibTwoRings}.6 & \ch{NH} & O & \ch{CH_3} & 5.5592 & 235\\ \hline
        TwoRings & \ref{figureCelecoxibTwoRings}.7 & O & \ch{NH} & \ch{CH_3} & -10.6712 & 77\\ \hline
        TwoRings & \ref{figureCelecoxibTwoRings}.8 & O & O & \ch{CH_3} & -3.5413 & 142
    \end{tabular}
\end{table}

\begin{table}[H]
    \centering	
	\caption{Conditions and statistics for the computation of the results from tables (\ref{TableMainResults1}) and (\ref{TableMainResults2}).}
	\label{TableConditionsMainResults}
	\begin{tabular}{c|c}
		NumberOfTrees            & 250\\\hline
		Erased Percentatge       & 0.0\%\\\hline
		Splitting proportion     & 10.0\% for testing \\\hline
		minimumCorrelationFactor & 0.0\\\hline
		Number of descriptors    & 3040\\\hline
		Mean Squared Error       & 1202.70\\\hline
		$r^2$	                 & 0.90\\\hline
		True Positive            & 994\\\hline
		False Positive           & 8\\\hline
		True Negative            & 1015\\\hline
		False Negative           & 101\\\hline
		True Positive Rate       & 0.91\\\hline
		True Negative Rate       & 0.99\\\hline
		ClassificationAcuracy    & 0.94\\\hline
		MatthewsCorrelationFactor& 0.87
    \end{tabular}
\end{table}



\end{document}
